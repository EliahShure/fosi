\documentclass{scrreprt} 
\usepackage{fontspec}
\usepackage{url}
\hyphenation{ngerman}
\usepackage{csquotes}
\setlength\parindent{0pt}
\begin{document}

\subject{Protokoll}
\title{FoSi 2016.03}
\subtitle{Drittes Foodsharing Siegen Treffen}
\author{Protokollant: Esteban Shure}
\date{Montag, den 21.03.2016 von 19:30 bis 21:30\\Hackspace Siegen e.~V., Effertsufer 104, 57072 Siegen}
\maketitle
\newpage
\tableofcontents
\newpage

\chapter{Formalia}
\section{Begrüßung}
Um 19:30 fingen wir mit dem Treffen im Hackspace e.~V. an. 
\section{Protokollant}
Esteban Shure führt das Protokoll im Etherpad\footnote{\url{https://etherpad.siegen.so/p/2016-03-21_FoSi2016.03}} und bittet um kollaborative Teilnahme. Das Protokoll wird über die Mailingliste \url{foodsharing@siegen.so}\footnote{Abonnieren: E-Mail an \url{foodsharing-subscribe@siegen.so}} verschickt.
\section{Anwesenheit}
Anwesend sind neben dem Protokollanten: 
\begin{enumerate}
	\item Malou
	\item Steffi
	\item Willi
	\item Gudrun
	\item Tobias B.
\end{enumerate}
Verspätet
\begin{enumerate}
	\setcounter{enumi}{5}
	\item Eliah Shure 19:37
	\item Marleen 19:37
	\item Benni 19:37
	\item Rakete 19:59
\end{enumerate}
Früher gegangen
\begin{itemize}
	\item Steffi 19:45
	\item Tobias 20:42
\end{itemize}
\newpage
Entschuldigt abwesend (Wann bekannt gegeben)
\begin{itemize}
	\item Philip E (2016-03-14)
    \item Adam (Vortag abend)
    \item Benita, (Vortag abend)
	\item Anne S., (11:57)
	\item Rebecca S., (15:15)
	\item Christina I, (19:08)
	\item Annika M, (19:14)
\end{itemize}

\section{Nächstes Treffen am Montag, den 04.04.2016 um 19:30 in den Vereinsräumen des Hackspace Siegen e.~V.}
    Bitte meldet euch vorzeitig ab. Arbeitsgruppen bitte frühzeitig Berichte einreichen. Aufgabenstati dürfen ebenfalls frühzeitig bekannt gegeben werden \footnote{\url{https://etherpad.siegen.so/p/2016-04-04_FoSi2016.04}}. 
    \begin{itemize}
    	\item Philip befindet sich bis zum 12.04. in Rotterdam.
    	\item Botschafter-Treffen im Mumble am 7.4.2016
    \end{itemize}

\chapter{Foodsharing e.~V.} Kurze Berichte der Botschafter über Neuigkeiten aus dem Verein.\par

\section{Fairteiler in Berlin (Philip)}
	Nichts Neues.
    
\chapter{Foodsharing in Siegen} Kurze berichte aus den Arbeitsgruppen über Geschehenes.

\section {AG Öffentlichkeitsarbeit}

\subsection{Infostand im Foyer am 14.04. (Steffi)}
 Am 14.04. ist eine kleine Aktion an der Uni Siegen geplant: Ein Infostand vor der Mensa des AR-Campus. Es wäre schön, wenn man vom Vortrag gerettete Lebensmittel dort verteilen könnte. Die Universitätsverwaltung hat sich noch nicht gemeldet und wird dies voraussichtlich im Lauf der nächste Woche tun. Willi kann am 14.04. gerettetes Essen aus Frankfurt mitnehmen. Flyer für den Infostand werden benötigt.
 
\subsection{Eigene Flyer gestalten (Steffi)} 
Die Gestaltung eigener Flyer ist noch nicht weiter fortgeschritten. Steffi trägt ihre Flyeridee vor, Teilnehmer besprechen diese detailiert. Es wird beschlossen, weitere Diskussion auf ein eigenes Treffen zu verschieben um dass sich die AG kümmert. Man ist sich einig, dass man lieber einen Flyer verteilt, als ewig am perfekten Flyer arbeitet. Auf dem letzten Treffen haben sich für das Erstellen eines Foodsharing Siegen Flyers Tobias, Steffi, Rebecca, Malou und Serious Lee interessiert. Mindestens ein Draft ist zu entwerfen  und an Adam oder einen anderen Layouter zu schicken.

\subsection{Aufgabenzustand}
Aufgaben von 2016-03-14:
	\begin{enumerate}
		\item Petition für Fairteiler veröffentlichen: Unerledigt
		\item Petition \enquote{Leere Tonne} veröffentlichen: Unerledigt
		\item Überprüfen, ob Interessierte schnell den Weg zur Plattform und Mailingliste finden: Unerledigt.
		\item Adam bitten, das Facebook-Cover so zu verändern, dass Kontaktdaten drin stehen? (Philip): Unerledigt. 
		\item Teaser und Foto für die Gruppe auf der Plattform erstellen und posten: Unerledigt.
		\item Kickstartertext uns mitteilen (Philip): Unerledigt.
	\end{enumerate} 
Kommende Aufgaben:
	\begin{enumerate}
		\item Nächstes Treffen als Event in Facebook veröffentlichen
		\item Philip hat einen Termin beim Bürgermeister am 12.04. um 13:00. 
		\item Kontakt zur Siegener Tafel aufbauen
		\item Kontakt zu lokalen Zeitungen und Radiosendern aufbauen
	\end{enumerate}

\section{AG Fairteiler}
\subsection{Weitere Fairteiler}
Weitere Orte für Fairteiler sind Scoutopia und das Schellack. Gudrun schlägt einen Fairteiler auf dem Gelände der Bluebox vor und wird dort anfragen.
\subsection{Aufgabenzustand}
Aufgaben vo 2016-03-14:
	\begin{itemize}
		\item Greenspace Fairteiler auf der Plattform eintragen: Unerledigt
		\item Dach des Schuppens vom Fairteiler Greenspace ausbessern und Gärtner ansprechen: Unerledigt.
		\item Teaser und Foto für die Gruppe auf der Plattform erstellen und posten: Unerledigt
	\end{itemize}
	
\section{AG Koordination}
Esteban bittet alle das Foodsaverquiz zu machen. Foodsaver bieten ihre Hilfe an. Darüber hinaus bittet Esteban sich auf der Plattform zu vernetzen, also \enquote{sich gegenseitig zu kennen} und zu \enquote{vertrauensbananisieren}.\par
Das aktuelle Treffen und notwendigen Unterlagen (Protokoll, Pad mit Tagesordnung\footnote{\url{https://etherpad.siegen.so/p/2016-03-21_FoSi2016.03}}, Termin im Veranstaltungskalender vom Hackspace Siegen e.~V.) wurden organisiert. Teaser-Text und Foto für die Gruppe auf der Plattform wurden erstellt. Es gibt auf der Plattform keinen allgemeinen Ort um die Bitte zum Eintragen auf die Mailingliste zu platzieren. Dafür ist der Hinweis nun in das Protokolltemplate übernommen worden. Zur Versicherungsfrage hat Esteban eine E-Mail über den Verteiler geschickt, mit der Bitte den Anhang zu lesen und gegebenenfalls zusammen zu fassen. 
 
\section{AG Kooperation}
Das Ergebnis des Treffens am Sonntag, den 23.03.2016 ist über die Mailingliste verschickt und vorgestellt worden. Die Bitte die Siegener Tafel anzusprechen ist an Philip weitergeleitet worden (bestehende Aufgabe). Eine Zusammenarbeit mit der Tafel führt dazu, dass man die Frage \enquote{Macht ihr denn nicht der Tafel Konkurrenz?} auch eine ordentliche Antwort geben kann \enquote{Wir arbeiten sogar mit denen!}. Teasertext und Foto für die Arbeitsgruppe auf der Plattform sind erstellt worden. Willi, Marleen und Benni berichten von ihren Abholererfahrungen, auch dass in kleinen Läden nur Mittwoch und Samstags abgeholt wird. Esteban, der nicht wusste, wie mit Einführungsabholungen umzugehen ist, wurde von ihnen entsprechend informiert. Siegener Foodsharer dürfen gerne in andere Bezirke fahren und beim Abholen helfen.\par
Vorschläge für weitere Kooperationen bitte über die Mailingliste schicken. Gudrun schlägt die kleine Bäckerei in Eiserfeld und im Herrengarten den Obst- und Gemüsehändler vor.\par 
Die Liste der Betriebe auf der Plattform wird gelöscht und neu eingetragen.\par
Der erste Betrieb soll angesprochen werden.
\chapter{Sonstiges}
Nichts

\section{Kurz notiert!}
Nichts
 
\chapter{Verteilte Aufgaben}
    \section{Alle}
    \begin{enumerate}
    	\item 2016-06-14: Beim Bestücken des Verteilers ein Bild an die Mailingliste, mindestens einen Text schicken.
		\item 2016-06-14:  Unser Facebook Community "Foodsharing Siegen" pushen durch Kommentare und liken\footnote{\url{https://www.facebook.com/foodsharingsiegen}}
		\item 2016-06-14: Unterstütze die Aktion "Leere Tonne" mit deiner Unterschrift
		\item 2016-06-14: Unterstütze die Aktion "Fairteiler Berlin" mit deiner Unterschrift
		\item 2016-06-14: Interessante Artikel und Aktionen an AG ÖA schicken.
    	\item 2016-06-14: Ersteban bietet an, Mumble einzurichten um über die Plattform zu kommunizieren.
    \end{enumerate}
    
    \section{Jemand}
    \begin{enumerate}
    	\item 2014-06-21: Versicherungsanhang lesen und zusammenfassen.
    \end{enumerate}
    
    \section{AG Öffentlichkeitsarbeit}
    \begin{enumerate}
   		\item 2016-03-14: Petition für Fairteiler veröffentlichen: Unerledigt
   		\item 2016-03-14: Petition \enquote{Leere Tonne} veröffentlichen: Unerledigt
   		\item 2016-03-14: Überprüfen, ob Interessierte schnell den Weg zur Plattform und Mailingliste finden: Unerledigt.
   		\item 2016-03-14: Adam bitten, das Facebook-Cover so zu verändern, dass Kontaktdaten drin stehen? (Philip): Unerledigt. 
   		\item 2016-03-14: Teaser und Foto für die Gruppe auf der Plattform erstellen und posten: Unerledigt.
   		\item 2016-03-14: Kickstartertext uns mitteilen (Philip): Unerledigt. 
    	\item 2016-03-14: Flyer für den Infostand.
  		\item 2016-03-14: Kontakt zur Siegener Tafel aufbauen
   		\item 2016-03-14: Kontakt zu lokalen Zeitungen und Radiosendern aufbauen
   		\item 2016-03-21: Nächstes Treffen als Event in Facebook veröffentlichen 
    \end{enumerate}  		
    
    \section{AG Fairteiler}
    \begin{enumerate}
   		\item 2016-04-21: Greenspace Fairteiler auf der Plattform eintragen
   		\item 2016-04-21: Dach des Schuppens vom Fairteiler Greenspace ausbessern und Gärtner ansprechen
   		\item 2016-04-14: Teaser und Foto für die Gruppe auf der Plattform erstellen und posten: Unerledigt
    \end{enumerate}
    
    \section{AG Koordination}
    \begin{enumerate}
    	\item 2016-06-21: Nächstes Treffen vorbereiten (TOPs, Einladung, Kalender)
       	\item 2016-06-21: Ausweise drucken, laminieren und verteilen
       	\item 2016-03-14: Botschafterbegrüßungsteam anschreiben, damit unsere Botschaftskandidaten schneller Botschafter werden: Esteban hilft dabei
    \end{enumerate}
    
    \section{AG Koorporation}
    \begin{enumerate}
    	\item 2016-06-21: Betriebe ansprechen
    	\item 2016-06-21: Betriebe auf der Plattform aktualisieren
    \end{enumerate}
        
    \section{Philip E.}
    \begin{enumerate}
    	\item 2016-03-14: Der Werbetext eines Kickstarters, welches du erwähntest, bitte in den Google-Ordner ablegen. 
    	\item 2016-03-14: Philip E-Mail von Dörthe an Benita
    \end{enumerate}
    
    \section{Rebecca S.}
    \begin{enumerate}
    	\item 2016-03-14: Nachricht an Esteban über die Plattform schicken
    \end{enumerate}
    
    \section{Gudrun}
    \begin{enumerate}
    	\item 2016-03-21: Bluebox Fairteiler
    \end{enumerate}

\end{document}

