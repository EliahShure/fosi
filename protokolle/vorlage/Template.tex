\documentclass{scrreprt} 
\usepackage{fontspec}
\usepackage{url}
\hyphenation{ngerman}
\usepackage{csquotes}
\usepackage{enumitem}
\setlength\parindent{0pt}
\begin{document}

\subject{Protokoll}
\title{FoSi 2016.04}
\subtitle{Viertes Foodsharing Siegen Treffen}
\author{Protokollant: Esteban Shure}
\date{Montag, den 04.04.2016 von 19:30 bis 20:50\\Hackspace Siegen e. V., Effertsufer 104, D-57072 Siegen}
\maketitle
\newpage
\tableofcontents
\newpage
\section{Nächstes Treffen am 11.04.2016 um 19:30 im Scoutopia e.~V., Weidenauer Straße 167, D-57076 Siegen}
	\begin{itemize}
		\item Pünktlich da sein, heißt um 19:15 herum.
		\item Jeder liest sich das Protokoll durch.
		\item Verteilte Aufgaben am Ende des Protokolls.
	\end{itemize}
\chapter{Formalia}
\section{Begrüßung}
Esteban begrüßt die Anwesenden um 19:30.
\section{Protokollan}
Esteban Shure führt das Protokoll im Etherpad\footnote{\url{https://etherpad.siegen.so/p/2016-04-04_FoSi2016.04}} und bittet um kollaborative Teilnahme.
Das Protokoll wird über die Mailingliste foodsharing@siegen.so\footnote{Abonnieren: E-Mail an foodsharing-subscribe@siegen.so} verschickt und im Github\footnote{\url{https://github.com/eshure/fosi/tree/master/protokolle}} archiviert.
\section{Anwesenheit}
Anwesend sind neben dem Protokollanten:
\begin{enumerate}
\item Eliah
\item Darren
\item Susanne
\item Willi
\item Sandra
\item Kevin
\item Annika
\end{enumerate}
Verspätet
\begin{enumerate}[resume]
\item Malte 19:34
\item Simon 19:34
\item Lukas, genannt \enquote{Zwiebel} 19:34
\item Becci 19:35
\item Malou 19:35
\item Oke 19:40
\item Tobias: 19:40
\item Gudrun 19:45
\item Katrin 19:53
\item Enis 19:54
\item Elizabeth 19:54
\end{enumerate}
Früher gegangen: niemand
Entschuldigt abwesend (Info; Wann bekannt gegeben)
\begin{itemize}
\item Benita (bis 11.04; 13:39)
\item Philip (bis 12.04; 21.03.2016)
\item Lars (bis einschließlich KW 14; 21.03.2016)
\end{itemize}

\chapter{Foodsharing e. V.}
Kurze Berichte der Botschafter über Neuigkeiten aus dem Verein. 
\section{Fairteiler in Berlin}
Petition auf unserer Facebookseite bekannt gemacht. Willi hatte an die Behörde einen Brief geschrieben und erhielt einen überheblich geschriebene Antwort. Er schickt den Brief per E-Mail an Esteban zur Weitergaben.
\section{BOT-Mumble}
Am 07.04.2016 um 19:00 findet ein Botschafter-Mumble statt auf dem Eliah Esteban vertreten wird. Seine Aufgabe wird sein, uns interessantes vom Treffen zu berichten.
\chapter{Foodsharing in Siegen}
Kurze Berichte aus den Arbeitsgruppen über Geschehenes.
\section{AG Öffentlichkeitsarbeit}
\subsection{Infostand}
\textbf{Am Donnerstag, den 14.04. findet zwischen 10 bis 14 Uhr der Infostand vor der Mensa im AR-Gebäude der Universität Siegen statt.} Es gibt folgende Auflagen: 
\begin{enumerate}
\item Auflagen: Nur 2 Personen am Infostand. 
\item Die Fläche beträgt maximal $ 1,5m \cdot 1,5m$. 
\item Stühle und Tische sind selber zu organisieren. 
\item Es darf nichts liegen gelassen werden oder Entsorgungskosten entstehen. 
\item Verteilen von Getränken oder Nahrungsmitteln benötigt eine Ausnahmeerlaubnis.
\end{enumerate} 
Becci will sich um eine Ausnahmeerlaubnis kümmern und hofft auf gerettete Lebensmittel für den Donnerstag, sonst gerne auch Containerware. Philip und Becci sind auf jeden Fall da, Becci organisiert über die Mailingliste einen Schichtplan. Außerdem werden am Donnerstag noch 75-er und 25-er Buttons gemacht, wobei Malou gerne hilft. Becci kommuniziert die Details über die Mailingliste. Die Materialkosten betragen 25 Cent, beziehungsweise 20 Cent und gehen an den Hackspace als Zurverfügungsteller der Buttonmaschine und -rohlinge.
\subsection{Flyer erstellen}
Steffi hat in Co-Creation einen neuen Flyer\footnote{\url{https://docs.google.com/document/d/1IjdH6EAYyGaAYshnoXGQUi9a2M4EpdMaBFtqBDmRI5U/edit}} gestaltet. 500 Stück werden übermorgen (06.04.2016) gedruckt, Lehrstuhl Bermann übernimmt die Kosten. Malou nimmt welche zum veganem Brunch mit, Esteban hätte gerne welche für den Erstkontakt. Interessierte melden sich bei Becci. 
\subsection{Flyer verteilen}
Becci organisiert beim nächsten Treffen, dass Darren das Verteilen am Wochenmarkt offiziel anmeldet. Esteban hilft gerne beim verteilen der Flyer. becci auf den Foodsaver treffen. Malte bietet sich an, die Mensa zu beflyern.
\subsection{Street Food Festival}
Becci berichtet, dass am Wochenende vom 22.05.(?) in Weidenau ein \enquote{Street Food Festival} statt findet, wozu wir thematisch gut passen. Es soll Standgebühren geben. Esteban denkt sich, dass man sich einen Stand sicherlich erschnorren kann. 
\subsection{Facebook-Coverbild}
Fotoschooting beim nächsten Termin im Scoutopia. Kleiderstil? Braun, weiß, grün, die Farben von foodsharing. Darren organisiert eine Einwillingserklärung für persönliche Bilder auf Facebook.
\subsection{Aufgabenzustand}
Erledigt:
\begin{itemize}
\item Petition für Fairteiler veröffentlichen:  (Facebook-Post am 25.03.2016)
\item Petition \enquote{Leere Tonne} veröffentlichen: (Facebook-Post am 31.03.2016)
\item  Überprüfen, ob Interessierte schnell den Weg zur Plattform und Mailingliste finden (dies könnte auf das Facebook Titelbild)
\item Adam bitten, das Facebook-Cover so zu verändern, dass Kontaktdaten drin stehen? (Philip) \footnote{(In Co-Creation: \url{https://docs.google.com/document/d/1djlQrG6PLXlvDwtX7z1ZO2nd-7_lGbSgY-6RFkJBz6k/edit}}
\item Kickstartertext uns mitteilen (Philip) (liegt im Google Drive Ordner : ÖffAr)
\item nächstes Treffen als Event in Facebook veröffentlichen (wird am 26.03 auf Facebook gepostet)
\end{itemize}
\section{AG Fairteiler}
\subsection{Greenspace, Effertsufer 34, D-57072 Siegen}
Annika berichtet, dass der Fairteiler im Greenspace aufgeräumt wurde. Lars kauft eine beschriftete Plane um mit dieser am Zaun auf den Fairteiler aufmerksam zu machen. Am Mittwoch, den 13.04.2016 trifft sich die AG um 17 Uhr für weitere Arbeiten am Fairteiler - Interessierte sind eingeladen.
\subsection{Bluebox, Sandstraße 54, D-57072 Siegen}
Gudrun hat mit Dieter Biermann und Rolf Schumann von der Blue Box gesprochen, die den Vorschlag ihrem Plenum vortragen. Es scheint eine Formalie zu sein, denn das Interesse war sehr hoch. Die Kontaktnummer schickt sie noch an Esteban, der diese an die AG Fairteiler weiter gibt. Tobias hat übrigens einen Kühlschrank für einen Fairteiler.
\section{AG Koordination}
Esteban hat Ausweise laminiert und reicht sie rum. Eliah erinnert daran, dass es nun nicht mehr notwendig ist die Rechtsvereinbarung\footnote{\url{http://wiki.lebensmittelretten.de/Rechtsvereinbarung}} zu unterschreiben, da das Bestehen zum Foodsaver auf der Plattform dazu ausreicht. 
\section{AG Kooperation}
Esteban hat Erstkontakt mit Aceyla Markt Lebensmittel in der Koblenzer Straße 178, D-57072 Siegen. Details auf der Plattform\footnote{\url{https://foodsharing.de/?page=fsbetrieb&id=14949}}. Foodsaver bitte für den Betrieb bewerben. Es fällt wohl einmal in der Woche eine Kiste von 3 bis 5 kg an.  
\chapter{Kurz notiert!}
Esteben war sprachlos, ob der Unpünktlichkeit von vielen. Dankenswerter Weise übernahm Becci das Willkommenheißen und die Vorstellung von Foodsharing.\\
Die WG überhalb vom \enquote{Der Biomarkt} war da und berichtete.
\chapter{Sonstiges}
\section{Unterstützung des Bündnis \enquote{Siegen ist bunt}}
Foodsharing e.~V. nimmt Abstand von allen politischen Aktionen, die nicht Lebensmittelverschwendung zum Thema haben. Deshalb können wir als \enquote{Foodsharing Siegen} nicht unter den Unterstützern auftreten. Nichtsdestoweniger dürfen wir als Privatpersonen am 23.04.2016 ab 12 Uhr mit zur Demo.
\chapter{Verteilte Aufgaben}
Alle
\begin{enumerate}
\item 2016-03-14: Beim Bestücken des Verteilers ein Bild an die Mailingliste, mindestens einen Text schicken.
\item 2016-03-14: Unser Facebook Community ”Foodsharing Siegen” pushen durch Kommentare und Likes\footnote{\url{https://www.facebook.com/foodsharingsiegen}}
\item 2016-03-14: Unterstütze die Aktion \enquote{Leere Tonne” mit deiner Unterschrift}
\item 2016-03-14: Unterstütze die Aktion ”Fairteiler Berlin” mit deiner Unterschrift
\item 2016-03-14: Interessante Artikel und Aktionen an AG ÖA schicken.
\item 2016-03-14: Ersteban bietet an, Mumble einzurichten um über die Plattform zu kommunizieren.
\item 2016-04-04: Treffen der AG Fairteiler am Mittwoch, den 13.04.2016 um 17 Uhr im Greenspace Siegen, Effertsufer 34 (?), D-57072 Siegen
\end{enumerate}
Jemand
\begin{enumerate}
\item 2014-03-21: Versicherungsanhang lesen und zusammenfassen.
\end{enumerate}
AG Öffentlichkeitsarbeit
\begin{enumerate}
    \item 2016-03-21: Teaser und Foto für die Gruppe auf der Plattform erstellen und posten
    \item 2016-03-21: Philip hat einen Termin beim Bürgermeister am 12.04. um 13:00.
    \item 2016-03-21: Kontakt zur Siegener Tafel aufbauen (wurde 2x angeschrieben, keine Rückmeldung). Diese erhalten wohl mehr, als sie weitergeben können.
    \item 2016-03-21: Kontakt zu lokalen Zeitungen und Radiosendern aufbauen.
    \item 2016-03-21: Nächstes Treffen als Event in Facebook veröffentlichen
    \item 2016-04-04: Malte ansprechen und Mensa beflyern.
    \item 2016-04-04: Pressemitteilung erster Fairteiler
\end{enumerate}
AG Fairteiler
\begin{enumerate}
	\item 2016-03-14: Teaser und Foto für die Gruppe auf der Plattform erstellen und posten:
	\item 2016-03-21: Greenspace Fairteiler auf der Plattform eintragen
	\item 2016-03-21: Dach des Schuppens vom Fairteiler Greenspace ausbessern und Gärtner ansprechen
	\item 2016-04-04: Einladen zum Treffen am 13.04.2016.
\end{enumerate}
AG Koordination
\begin{enumerate}
	\item 2016-03-21: Nächstes Treffen vorbereiten (TOPs, Einladung, Kalender, Archiv)
	\item 2016-03-21: Ausweise drucken, laminieren und verteilen
	\item 2016-04-04: Begrüßungsteam schreiben, wer neu beigetreten ist.
\end{enumerate}
AG Koorporation
\begin{enumerate}
	\item 2016-04-21: Betriebe ansprechen
	\item 2016-04-21: Betriebe auf der Plattform aktualisieren
\end{enumerate}
Philip E.
\begin{enumerate}
    \item 2016-03-14: Philip E-Mail von Dörthe an Benita
    \item 2016-04-04: Schichtleitung Infostand am Infostand.
\end{enumerate}
Becci
\begin{enumerate}
    \item 2016-04-04: Ausnahmeerlaubnis für Nahrungsmittel für den Infostand.
    \item 2016-04-04: Schichtleitung am Infostand
    \item 2016-04-04: Schichtplanorganisation für den Infostand
    \item 2016-04-04: Buttons machen 
    \item 2016-04-04: Verteilen von Flyern beim nächsten Treffen organisieren
\end{enumerate}
Darren
\begin{enumerate}
	\item 2016-04-04: Einwilligungserklärung für Gemeinschaftsfoto auf der Facebookcoverseite
\end{enumerate} 
Eliah
\begin{enumerate}
    \item 2016-04-04: Esteban beim BOT-Mumble am 7.4. um 19 Uhr vertreten und Infos weitergeben.
\end{enumerate}    
Gudrun
\begin{enumerate}
	\item 2016-04-04: Bluebox Fairteiler Kontakt an Esteban schicken 
\end{enumerate}
Malou
\begin{enumerate}
    \item 2016-04-04: Buttons machen
    \item 2016-04-04: Flyer zum veganen Brunch mitnehmen
\end{enumerate}
Steffi
\begin{enumerate}
	\item 2016-04-04: Flyer drucken
\end{enumerate}
Willi
\begin{enumerate}
	\item 2016-04-04: Esteban den Antwortbrief der Berliner Behörde schicken.
\end{enumerate}
    

\end{document}

