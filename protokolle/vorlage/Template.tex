\documentclass{scrreprt} 
\usepackage{fontspec}
\usepackage{url}
\hyphenation{ngerman}
\usepackage{csquotes}
\setlength\parindent{0pt}
\begin{document}

\subject{Protokoll}
\title{FoSi 2016.02}
\subtitle{Zweites Foodsharing Siegen Treffen}
\author{Protokollant: Esteban Shure}
\date{Montag, den 14.03.2016 von 19:34 bis 20:43\\Hackspace Siegen e.~V., Effertsufer 104, 57072 Siegen}
\maketitle
\newpage
\tableofcontents
\newpage

\chapter{Formalia}
\section{Begrüßung}
Um 19:34 fingen wir mit dem Treffen im Hackspace e.~V. an. 
\section{Protokollant}
Esteban Shure führt das Protokoll im Etherpad\footnote{\url{https://etherpad.siegen.so/p/2016-03-14_foodsharing}} und bittet um kollaborative Teilnahme. Das Protokoll wird über die Mailingliste \url{foodsharing@siegen.so} verschickt.
\section{Anwesenheit}
Anwesend sind neben dem Protokollanten: 
\begin{enumerate}
	\item Philip E.,
	\item Steffi M. Christina B.,
	\item Rebecca S. Benita, 
	\item Malou, 
	\item Eliah Shure,
	\item Tobias B.,
	\item Serious Lee and
	\item Annika
\end{enumerate}
Verspätet, weil Esteban nicht ans Handy geht: Gudrun (19:47). Entschuldigt abwesend:
\begin{enumerate}
	\item Lars Tubies befindet sich auf einer Insel
	\item Adam befindet sich an einem Arbeitstisch
	\item Anne Schwab befindet sich im Jet Lag
\end{enumerate}
\section{Nächstes Treffen am Montag, den 21.03.2016 um 19:30 im HaSi.}
    Philip befindet sich bis zum 12.04. in Rotterdam.

\chapter{Foodsharing e.~V.}
Kurze Berichte der Botschafter über Neuigkeiten aus dem Verein.\par

\section{Esteban Shure: Botschafter-Mumble vom 2016-03-07}
\begin{itemize}
	\item Betriebsketten: Wir sollen auf keinen Fall Betriebsketten und Franchise selber ansprechen, weil
	\begin{itemize}
		\item die Betriebsketten sich um Imageverlust und Arbeitsaufwand fürchten
		\item die Marke foodsharing sonst chaotisch wirkt (jede Stadt fragt die Kette an)
		\item es für jede spendende Betriebskette einen KAM (Key Account Manager) gibt, die unsere Schnittstelle darstellt.
	\end{itemize}
	\item Spendende: Nur Botschafter sollen die Liste über den aktuellen Zustand der Betriebsketten einsehen. Also fragt eure Botschafter. Spendende Betriebe sind: Kaufland und Backwerk. 
	\item Aufwand: Betriebsketten bedeuten hohen Arbeitsauwand. Lieber in einem kleinen Laden 100\% abholen als in einem Großen qualitativ nur 50\% stemmen können.
	\item AG Koorperation: Checkliste für Kooperationsaufwand und -pflege \footnote{\url{http://wiki.lebensmittelretten.de/Checkliste_Kooperationsaufbau_und_Kooperations\-pflege}}
	\item AG Fairteiler: Artikel über einen Fairteiler in Berlin \footnote{\url{http://www.berlin.de/sen/verbraucherschutz/aufgaben/gesundheitlicher-verbraucherschutz/lebensmittelretter/artikel.445869.php}}
\end{itemize}
	\section{Fairteiler}
	Philip berichtet, dass er kein Update hat, fragt aber weiterhin nach.     Es bleibt spannend. Die Petition soll auf unsere Facebook-Seite um Unterschriften zu sammeln. Auch für die Aktion \enquote{Leere Tonne}.
    Philip berichtet von seinen Erfahrungen des \enquote{Hilfe, man will mich wegschmeissen}-Ereignisses in Chemnitz.

\chapter{Foodsharing in Siegen}
Kurze berichte aus den Arbeitsgruppen über Geschehenes.
\section {AG Öffentlichkeitsarbeit}
\subsection{Philp: Bürgermeister, Presse, Siegener Tafel}
Philip hat einen Termin beim Bürgermeister am 12.04. um 13:00. Er berichtet ihm dann über Foodsharing und Yunity\footnote{\url{https://project.yunity.org/}}. Im Vorfeld hatte er über den Hackspace Siegen\footnote{\url{https://www.hasi.it}} und Scoutopia \footnote{\url{https://www.facebook.com/Scoutopia-339876692877996/}} berichtet. Außerdem hat er die Siegener Tafel \footnote{\url{www.siegener-tafel.de}} ein zweites Mal kontaktiert. Darüber hinaus hat er lokale Zeitungen und Radiosender angeschrieben.
\subsection{Rebecca: Mailingliste}
Rebecca ist nun Moderatorin der Mailingliste \url{foodsharing@siegen.so}. Wer auf der Mailingliste ist, kann direkt an die Liste schreiben. Beiträge von außerhalb werden von Rebecca moderiert. Sollte ein Listenmitglied sich dabenen benehmen, bannt Rebecca ihn von der Liste. Die Mailingliste ist für Ankündigungen und den Austausch von nicht-zeitkritischen Informationen gedacht. Daneben bleibt Benita unsere Telefonzentrale, es ist wichtig, dass jeder von uns Benitas Nummer hat, damit wir im kurzfristig Kontakt untereinander aufbauen können (Vor Allem später beim Lebensmittel abholen).
\subsection{Philip: Facebook}
Die Gruppe hat mehr als 500 Likes und 20 mehr in dieser Woche. Der Vorschlag like-ende Personen mit einem Text anzuteasern ist als nicht attraktiv genug für Neuinteressenten bewertet worden. Der Facebookauftritt soll auf überzeugende Weiterleitung in unser Netzwerk überprüft werden.
\subsection{Eliah Shure: Flyer}
Es sind ca. 40 Flyer angekommen und diese sollen so schnell wie möglich verteilz werden. Ihr Nachdruck ist Seitens foodsharing.de geplant. 
\subsection{Rebecca: Infostand}
Rebecca berichtet, dass sie einen Infostand für den 14.04.2016 angemeldet hat. Eine Dame wurde kontaktiert und über foodsharing informiert. Philip berichtet von einem Kickstarter, welcher guten Werbetext enthält. 
\section{AG Fairteiler}
\subsection{Malou: Neuer Fairteiler im Greenspace Siegen}
Der Greenspace hat nun einen Fairteiler. Lars T. wohnt nebenan, es gibt ein Regal, keinen Kühlschrank. Er muss noch auf der Plattform eingetragen werden. Bilder sollen auf die Mailingliste geschickt werden, damit AG ÖA diese verwerten kann und Listenmitglieder einen schnellen Überblick über den aktuellen Zustand gewinnen können. Auch später, wenn Foodsaver den Fairteiler beliefern wird um Fotos oder mindestens ein Text gebeten.
\section{AG Koordination}
\subsection{Esteban: Plattform-Vernetzung}
Bitte klickt euch gegenseitig \enquote{Ich kenne X} an und verschenkt Verrauensbananen, damit wir innerhalb der Plattform als aktiver Bezirk erkannt werden, nachdem wir 2 Jahre inaktiv waren.\par
Bitte besteht mindestens das Foodsaver-Quiz, bewerbt euch für interessante Arbeitsgruppen und bringt euch ein. Bei Fragen Esteban anschreiben - über die Plattform. 
\section{AG Kooperation}
Die AG nimmt Vorschläge entgegen, welche Betriebe angesprochen werden sollen. Bei ihrem Treffen wird eine Liste erstellt, das Treffen ist öffentlich.
\begin{itemize}
	\item \enquote{Der Biomarkt} ansprechen
	\item \enquote{Ali Ceylan} (?), wegen Nähe zum Fairteiler Greenspace
	\item \enquote{Benitas} Gemüsehändler, wegen Nähe zum Fairteiler Scoutopia
	\item \enquote{Suppenküche} von Barbara McCoy
	\item eine Bäckerei finden. 
	\item einen Laden in der Oberstadt finden, weil dort viele Foodsaver sind
	\item Picknicker anfragen
	\item Sakura anfragen
	\item Gemüseändler Enders in Eiserfeld fragen
\end{itemize}


\chapter{Sonstiges}
\section{Rebecca: Info-Stand an der Uni am 14.04.2016 von 10:00 - 15:00}
	Rebecca berichtete von dem Gespräch mit Maria Schmidt (?). Diese Woche wird geklärt, ob der Termin für den Infostand funktioniert. Zwischen 10 und 15 Uhr soll ein Stand vor der Uni-Mensa stehen. Flyer und Buttons sollen Interessierten in die Hand gedrückt werden. Es wäre schön, wenn soeben gerettetes Essen verteilt werden könnte. Notfalls containerte Lebensmittel. AG ÖA kümmert sich um Flyer. Bitte bereitet euch auf Infostandschichten in diesem Zeitraum vor. 
\section{Kurz notiert!}
	\begin{itemize}
		\item Botschafterbegrüßungsteam anschreiben, damit unsere Botschaftskandidaten schneller Botschafter werden: Esteban hilft dabei
		\item Tobias berichtet von der Kommerzialisierung von Infoständen, Monitorzeiten und Werbeflächen. 
		\item Christina berichtet, dass an der Uni Darmstadt der AStA einen Raum für foodsharer zur Verfügung steht. 
		\item Tobias berichtet, dass der AStA für Initiativen zuständig ist. Hygieneschein haben Gudrun und Annika. Flyer in der Mensa gehört zum StudWerk, Platz vor der Mensa gehört in den Verantwortungsbereich von anderer Organisation. Flyer vor der Schranke verteilen ist in Ordnung.
		\item Tobias warnt vor Hygieneanforderungsprobleme. Annika und Gudrun haben so eine Bescheinigung.
		\item Malou berichtet von Grünen / Blauen Engel, die ebenfalls Lebensmittel annehmen und verteilen.	
		\item Für das Bestehen des Foodsaver-Quiz' empfiehlt Tobias den Zeit begrenzten Text und Eliah in der Nähe dabei zu haben.	
	\end{itemize}
 
\chapter{Verteilte Aufgaben}
    \section{Alle}
    \begin{enumerate}
    	\item Beim Bestücken des Verteilers ein Bild an die Mailingliste, mindestens einen Text schicken.
		\item Unser Facebook Community "Foodsharing Siegen" pushen durch Kommentare und liken\footnote{\url{https://www.facebook.com/foodsharingsiegen}}
		\item Unterstütze die Aktion "Leere Tonne" mit deiner Unterschrift
		\item Unterstütze die Aktion "Fairteiler Berlin" mit deiner Unterschrift
		\item Interessante Artikel und Aktionen an AG ÖA schicken.
    	\item bietet an, Mumble einzurichten um über die Plattform zu kommunizieren.
    \end{enumerate}
    \section{AG Öffentlichkeitsarbeit}
    \begin{enumerate}
    	\item Petition für Fairteiler veröffentlichen
    	\item Petition \enquote{Leere Tonne}
    	\item Überprüfen, ob Neuinteressierte schnell den Weg zu uns, der Mailingliste und der Plattform finden.
    	\item Kann Adam das Facebook-Cover so verändern, dass Kontaktdaten drin stehen? (Philip).  
    	\item Erstellen eines foodsharing.de-Siegen Flyers. Interessierte für diese Aufgabe sind Tobias, Stefanie, Rebecca und Serious Lee. Mindestens ein Draft ist zu entwerfen und an Adam oder einen anderen Layouter zu schicken. 
    	\item Flyer für den Infostand.
    	\item Teaser und Foto für die Gruppe auf der Plattform erstellen und posten
    	\item Unterschrift-Sammel-Aktionen posten
    \end{enumerate}
    \section{AG Fairteiler}
    \begin{enumerate}
    	\item Greenspace Fairteiler auf der Plattform eintragen.
    	\item Weitere Orte für Fairteiler: Scoutopia, Schellack
    	\item Dach des Schuppens vom Fairteiler Greenspace ausbessern?
      	\item Teaser und Foto für die Gruppe auf der Plattform erstellen und posten
    \end{enumerate}
    \section{AG Koordination}
    \begin{enumerate}
       	\item Teaser und Foto für die Gruppe auf der Plattform erstellen und posten
    \end{enumerate}
    \section{AG Koorporation}
    \begin{enumerate}
    	\item Greenspace Fairteiler auf der Plattform eintragen.
    	\item Weitere Orte für Fairteiler: Scoutopia, Schellack
       	\item Teaser und Foto für die Gruppe auf der Plattform erstellen und posten
    \end{enumerate}
    \section{Esteban Shure}
    \begin{enumerate}
    	\item Haftpflichtversicherung-Thread schicken.
    	\item Treffen organisieren
    	\item Einführungsabholungen - wie? (Esteban)
    	\item Engelsystem.de für Infostand- und Abholeraufgaben?
    \end{enumerate}
    \section{Philip E.}
    \begin{itemize}
    	\item Der Werbetext eines Kickstarters, welches du erwähntest, bitte in den Google-Ordner ablegen. 
    	\item Philip E-Mail von Dörthe an Benita
    \end{itemize}
    \section{Rebecca S.}
    \begin{itemize}
    	\item Nachricht an Esteban über die Plattform schicken
    \end{itemize}
 




\end{document}

