\documentclass{scrreprt} 
\usepackage{fontspec}
\usepackage{url}
\hyphenation{ngerman}
\usepackage{csquotes}
\usepackage{enumitem}
\setlength\parindent{0pt}
\usepackage{pdfpages}% http://ctan.org/pkg/pdfpages
\begin{document}

\subject{Protokoll}
\title{FoSi 2016.05}
\subtitle{Fünftes Foodsharing Siegen Treffen}
\author{Protokollant: Esteban Shure}
\date{Montag, den 11.04.2016 von 19:30 bis 21:00\\Scoutopia, Weidenauer STraße 167, D-57076 Siegen}
\maketitle
\newpage
\tableofcontents
\newpage
\section{Nächstes Treffen am 18.04.2016 um 19:30 im Scoutopia e.~V., Weidenauer Straße 167, D-57076 Siegen}
	\begin{itemize}
		\item Pünktlich da sein, heißt um 19:15 herum.
		\item Jeder liest sich das Protokoll durch.
		\item Deine Aufgaben stehen am Ende des Protokolls.
		\item Ab 19 Uhr Kennenlernen.
	\end{itemize}
\chapter{Formalia}
\section{Begrüßung}
Esteban begrüßt die Anwesenden um 19:31.
\section{Protokollant}
Esteban Shure führt das Protokoll im Etherpad\footnote{\url{https://etherpad.siegen.so/p/2016-04-11_FoSi2016.05}} und bittet um kollaborative Teilnahme.
Das Protokoll wird über die Mailingliste foodsharing@siegen.so\footnote{Abonnieren: E-Mail an foodsharing-subscribe@siegen.so} verschickt und im Github\footnote{\url{https://github.com/eshure/fosi/tree/master/protokolle}} archiviert.
\section{Anwesenheit}
Anwesend sind neben dem Protokollanten:
\begin{enumerate}
\item Gregor
\item Vivi
\item Svenja
\item Tobas
\item Steffi
\item Malou
\item Annika
\item Anne
\item Philip
\item Becci
\item Eliah
\item Darren
\item Nadia
\item Malte
\item Mira
\item Willi
\item Corinna
\end{enumerate}
Verspätet\\
Niemand (Esteban hat es nicht notiert)
%\begin{enumerate}[resume]
%\end{enumerate}
%\begin{itemize}
%\end{itemize}
Früher gegangen
\begin{enumerate}
\item Mira 20:31
\item Gregor 20:35
\item Vivi 20:35
\end{enumerate}
Entschuldigt abwesend (Info; Wann bekannt gegeben)
\begin{itemize}
\item Lars (bis einschließlich KW 14; 21.03.2016)
\item Sandra (nur heute, 12:00)
\end{itemize}
\chapter{Foodsharing e. V.}
Vorstellung und kurze Berichte der Botschafter über Neuigkeiten aus dem Verein. 
\section{Beantwortung von Fragen der Neuen zum Verein}
Wir stellten uns vor. Wir sprachen über unsere Idee und unsere lokale Struktur in Arbeitsgruppen. Wir erklärten die Rechtsvereinbarung, die Plattform \url{https://foodsharing.de} und die Besonderheit von Betriebsketten.\\
Achtung: Wir haben den Zettel mit den E-Mailadressen der Neuen nicht wieder eingesammelt. Rekonstruieren und auf die Mailingliste setzen konnte Esteban (Stand 2016-04-13): Svenja, Vivi und Gregor. Es fehlen Corinna (Willi?), Nadia (Malte?) und Mira (Jemand?). Raphael Fellmer hat Interesse unseren Bezirk zu besuchen und einen Vortrag zu halten. Dafür ist die Organisation eines Hörsaales notwendig.
\section{bundesweit}
Foodsharing hat den Smart-Hero Award gewonnen und die beiden Raphaels sind in Berlin um einen weiteren Preis abzuholen (\enquote{Young Idea Leaders Nachhaltigkeit Bla-Foo}). Philip berichtet, dass mittlerweile 3,5 Mio kg (oder Tonnen?) Lebensmittel gerettet wurden.
\section{Fairteiler in Berlin}
Weiterhin schwierige Situation, wir bitten die Neuen die Petition zu unterschreiben. Diese wurde erneut auf unserer Facebookseite erwähnt. Darren erklärt seinen Hack: Den Fairteiler zur Mülltonne deklarieren und mit Draht einfrieden. Georg berichtet von einem Bäcker in Dortmund, der dies so gelöst hat. Willi erzählt von der automatisch enthaltenen Antwort, wenn man die Petition unterschreibt. In Frankreich dürfen rechtlich keine Lebensmittel mehr weggeworfen werden. Diese werden nun von Entsorgungsfirmen aufgekauft, weiterverarbeitet zu Tierfutter oder dann entsorgt. 
\section{BOT-Mumble}
Am 07.04.2016 um 19:00 fand ein Botschafter-Mumble statt auf dem Eliah Esteban vertreten hat. Es gab dort nichts zu besprechen, zwei kleinere lokale Anfragen sind beantwortet worden. Das Treffen war dann ein informeller Austausch über Fahrräder. 
\section{Fahrräder}
Bei FlixBus konnte man Pfennige spenden und die gemeinnützige Organisation, die diese am Ende gewonnen hat, heißt foodsharing. Der Preis sind mehrere Lastenfahrräder um die man sich bewerben kann. Darum möchte sich die AG Öffentlichkeitsarbeit und Steffi kümmern.
\section{Veranstaltungen}
\begin{enumerate}
\item Donnerstag, den 14.04.2016 um 18:00 im Scoutopia: Vortrag, Care Revolution
\item Donnerstag, den 14.04.2016 um 20:30 im Hackspace Siegen: Vortrag, Yunity (Unser aller Philip Engelbutzeder)
\end{enumerate}

\chapter{Foodsharing in Siegen}
Kurze Berichte aus den Arbeitsgruppen über Geschehenes, Aktuelles und noch zu Erledigendes.
\section{AG Öffentlichkeitsarbeit}

\subsection{Eigenwerbung am Mittwoch, den 13.04.2016}
Nachtrag 2016-04-12 um 23:40: \\
Becci informierte am Dienstag, den 12.04.2016 um 17:14 mindestens die Telegram-Gruppe: \enquote{Hallo ihr lieben, kurze Info wegen der Flyeraktion morgen:\\
Wir machen 2 Gruppen. Ich werde von 10-13 Uhr unterwegs sein. Treffpunkt am Marktplatz um 10 Uhr. Philip wird ab 14:30 bis ende offen (ca. 18/19 uhr) unterwegs sein. Treffpunkt vor der Engelapotheke. Alle die mitkommen wollen, können einfach vorbei kommen zu den jeweiligen Treffpunkten. Wenn ihr später kommt, dann kontaktiert uns einfach telefonisch. Ich schick auch die Infos heute Abend nochmal per mail rum. Bis morgen, ich freu mich auf den Tag mit euch!  :)}\\
Flyer und Buttons in der Innenstadt verteilen. Um 16 Uhr Essen bei Darren möglich (Nachtrag: Er kündigt die Speiskarte an: Brühe in der Tasse mit aufgewärmten Containerbrot). Auftritt mit Buttons und schwarzen T-Shirts. Becci verschickt Infos über die Mailingliste.

\subsection{Infostand am Donnerstag, den 14.04.2016}
{Am Donnerstag, den 14.04. findet zwischen 10 bis 14 Uhr der Infostand vor der Mensa im AR-Gebäude der Universität Siegen statt. Es gibt folgende Auflagen: 

\begin{enumerate}
\item Auflagen: Nur 2 Personen am Infostand. 
\item Die Fläche beträgt maximal $ 1,5m \cdot 1,5m$. 
\item Stühle und Tische sind selber zu organisieren. Der Hausmeister ist im Vorfeld höflich um einen Tisch zu fragen.
\item Es darf Nichts liegen gelassen werden oder Entsorgungskosten entstehen. 
\item Verteilen von Getränken oder Nahrungsmitteln benötigt eine Ausnahmeerlaubnis - die wir, Dank Becci, erhalten haben
\end{enumerate} 
Becci hofft auf gerettete Lebensmittel für den Donnerstag, sonst gerne auch Containerware. Philip und Becci sind auf jeden Fall da, Becci organisiert über die Mailingliste einen Schichtplan. 

\subsection{StreetFoodFestival, Freitag, der 21. bis Samstag, der 22.05.2016 }
Wir bewerben uns um einen Stand. Allerdings kann es da kein gerettetes Essen geben. So \enquote{Street}-Food ist die Veranstaltung dann doch nicht. Esteban hat leider keine Informationen über Standgebühren notiert.

\subsection{Facebook-Coverbild}
Darren sollte eine Einwilligungserklärung organisieren. Wird er zum nächsten Treffen machen. Hintergrund, wer uneingewilligt auf Facebook dargestellt wird kann klagen, sagt Darren. \enquote{YOLO} ist ein lauter Gedanke im Kopfe des Protokollanten. Des weiteren sind zu wenige in den foodsharing Farben braun, weiß, grün erschienen. Ein Foto war für nach dem Treffen geplant, dann gingen aber schon die meisten. Also verschoben auf das nächste Treffen. Ein Coverbild für unsrer Facebook-Seite und Gruppe mit unseren Gesichtern darauf macht und persönlicher.\\
Vivi, Gregor und Tobias wollen sich als Obst oder Gemüse verkleiden. 

\subsection{Sonstiges}
Das Treffen mit dem Bürgermeister, so Philip, findet im Mai statt. Am Mittwoch erfolgt dennoch ein Treffen mit Frau Schotz von der Stadt. Weitere Fairteilerstandorte können sein KaffeeChaos und VEB. 

\section{AG Fairteiler}
Am Mittwoch, den 13.04.2016 traf sich die AG um 17 Uhr für weitere Arbeiten am Fairteiler. Darren berichtet, dass Anforderungen an einen Fairteiler Landesrecht sind.

\subsection{Greenspace, Effertsufer 34, D-57072 Siegen (Malou)}
Malou berichtet von hygienischen Problemen. Leute, die sich wie Arschlöcher verhalten, pinkeln in den Geräteschuppen. Drei Äpfel und ein Tortilla-Dip lagen im Regal. Es gibt keinen Foodsharer-Verkehr. Lars kauft eine beschriftete Plane um mit dieser am Zaun auf den Fairteiler aufmerksam zu machen. Corinna berichtet von einem Fairteiler in Frankfurt, den sie näher kennt. Dort sind es zum Regal zusammengenagelte Holzboxen in der freien Luft. Malou hält dies für ein gutes Konzept für den Greenspace. Darren befürchtet das Gesundheitsamt und, nicht vergessen, in der unmittelbaren Nachbarschaft, arbeitet eine, auch in ihrer Freizeit, achtsame Dame beim Ordnungsamt.

\subsection{Bluebox, Sandstraße 54, D-57072 Siegen (Annika)}
Esteban hat den Kontakt von Gudrun erhalten und an die Gruppe weitergeleitet. Annika berichtete, dass Herr Ralf Schuhmann (von der BlueBox) von der Idee nichts wusste. Er bringt es Mittwoch in das Plenum ein. Er glaubt nicht an den Erfolg dieser Idee, denn \enquote{Wir können ja nicht alles machen}. Gudrun hatte beim letzten Treffen berichtet mit Dieter Biermann und Rolf Schumann (beide von der Blue Box) gesprochen. Die damals angekündigte \enquote{kleine Formalität} und hohe Interesse werden sich nun zeigen. Die Anwesenden nahmen diese Wendung traurig auf, waren sichtlich geknickt. Eine beachtliche Mehrheit suchte Trost in den vielen süßen Teilchen, die Corinna und Willi gerettet und zum Treffen mitgebracht haben. 

\subsection{Effertsufer 104 (Darren)}
Darren möchte den Vermieter vom Effertsufer 104 anfragen, ob im Eingangsbereich nicht ein entsprechendes Regal aufgebaut werden kann.

\section{AG Koordination}
Esteban berichtet, dass er am 18.04. bereits in Köln arbeitet und nicht zum Treffen kommen kann. Darren, Eliah und Tobi finden sich am Mittwoch, den 13.04.2016 um 16 Uhr bei Darren ein um die AG Koordination zum Einen auf mehr Schultern zu verteilen und, seien wir mal ehrlich, zu professionalisieren. Esteban freut sich. Wir alle dürfen gespannt sein.

\section{AG Kooperation}
Esteban hat den Gemüsehändler Schneider in der City-Galerie ansprechen wollen. Allerdings ist dieser dort nicht mehr. Stattdessen steht dort nun das Salädchen, welches eine Betriebskette ist und Verhandlungen mit foodsharing.de laufen bereits. Die Liste der nächsten, anzusprechenden Betriebe ist nach dem damaligen Treffen der AG über die Mailingliste gegangen und hier nochmal im Anhang.\\
Esteban spricht die Tafel an, die AG Öffentlichkeitsarbeit hat nach zwei Anfragen nichts weiter unternommen. Willi kennt dort Leute persönlich, er wird diese inoffiziell anfragen und entweder Kontaktdaten an die AG Kooperation weitergeben oder die Absage.

\chapter{Kurz notiert!}
Esteban ist leicht verstimmt, weil die Arbeitsgruppen die Arbeit, die sie in separaten Treffen erledigen sollen, in das Treffen verlegen. Die Berichte arten so zu Diskussionen aus. Dies trifft ihn als Protokollant doppelt. Professionelle Berichte bestehen aus drei Teilen:
\begin{itemize}
\item Das haben wir erledigt
\item Das sind unsere aktuellen Aufgaben
\item Das wollen wir beim nächsten Mal erledigt haben
\end{itemize}
Des weiteren wird das Protokollpad früh genug bekannt gegeben. Dort kann man seine Berichte schon vor dem Treffen einpflegen. Genauso, wie jeder dort unter seiner Gruppe und seinem Namen, die ihm zugeteilten Aufgaben als erledigt markieren kann. Die AG Koordination möchte begleitend koordinieren und benötigt dazu fein-zeitlichere Rückmeldung von euch, als alle 7 Tage auf dem Treffen. Das ist zu ruckartig.
\chapter{Sonstiges}
Darren berichtet über Neues aus der aktuellen Biomarkt-Situation in Siegen.

\chapter{Verteilte Aufgaben}

Alle
\begin{enumerate}
\item 2016-03-14: Beim Bestücken des Verteilers ein Bild an die Mailingliste, mindestens einen Text schicken.
\item 2016-03-14: Unser Facebook Community \enquote{Foodsharing Siegen} pushen durch Kommentare und Likes\footnote{\url{https://www.facebook.com/foodsharingsiegen}}
\item 2016-03-14: Unterstütze die Aktion \enquote{Leere Tonne” mit deiner Unterschrift}
\item 2016-03-14: Unterstütze die Aktion \enquote{Fairteiler Berlin} mit deiner Unterschrift
\item 2016-03-14: Interessante Artikel und Aktionen an AG ÖA schicken.
\item 2016-03-14: Eliah bietet an, Mumble einzurichten um über die Plattform zu kommunizieren.
\item 2016-04-11: Flyer für den Erstkontakt erhaltet ihr nun bei der AG Öffentlichkeitsarbeit, genauer bei Becci.
\end{enumerate}


Jemand
\begin{enumerate}
\item 2014-03-21: Versicherungsanhang lesen und zusammenfassen. Siehe Abschnitt \ref{app:versichern}.
\end{enumerate}

AG Öffentlichkeitsarbeit
\begin{enumerate}
    \item 2016-03-21: Teaser und Foto für die Gruppe auf der Plattform erstellen und posten
    \item 2016-03-21: Philip hat einen Termin beim Bürgermeister am 12.04. um 13:00.
    \item 2016-03-21: Kontakt zur Siegener Tafel aufbauen (wurde 2x angeschrieben, keine Rückmeldung). Diese erhalten wohl mehr, als sie weitergeben können.
    \item 2016-03-21: Kontakt zu lokalen Zeitungen und Radiosendern aufbauen.
    \item 2016-03-21: Nächstes Treffen als Event in Facebook veröffentlichen
    \item 2016-04-04: Malte ansprechen und Mensa beflyern.
    \item 2016-04-04: Pressemitteilung erster Fairteiler
    \item 2016-04-11: Bewerbung für ein Lastenrad
\end{enumerate}

AG Fairteiler
\begin{enumerate}
	\item 2016-03-14: Teaser und Foto für die Gruppe auf der Plattform erstellen und posten
	\item 2016-03-21: Greenspace Fairteiler auf der Plattform eintragen
	\item 2016-03-21: Dach des Schuppens vom Fairteiler Greenspace ausbessern und Gärtner ansprechen
	\item 2016-04-04: Einladen zum Treffen am 13.04.2016.
	\item 2016-04-11: Fairteiler beim VEB anfragen
	\item 2016-04-11: Fairteiler beim KaffeeChaos anfragen.
\end{enumerate}
AG Koordination
\begin{enumerate}
	\item 2016-03-21: Nächstes Treffen vorbereiten (TOPs, Einladung, Kalender, Archiv)
	\item 2016-03-21: Ausweise drucken, laminieren und verteilen
	\item 2016-04-04: Begrüßungsteam (Benita) über die Plattform schreiben, wer neu beigetreten ist.
\end{enumerate}
AG Koorporation
\begin{enumerate}
	\item 2016-04-21: Betriebe ansprechen
	\item 2016-04-21: Betriebe auf der Plattform aktualisieren
\end{enumerate}
Becci
\begin{enumerate}
    \item 2016-04-04: Schichtleitung am Infostand
    \item 2016-04-04: Schichtplanorganisation für den Infostand
    \item 2016-04-04: Verteilen von Flyern beim nächsten Treffen organisieren
\end{enumerate}
Darren
\begin{enumerate}
	\item 2016-04-04: Wiedervorlage 1: Einwilligungserklärung für Gemeinschaftsfoto auf der Facebookcoverseite
	\item 2016-04-11: Fairteiler Effertsufer 104: Vermieter anfragen.
	\item 2016-04-11: Mittwoch, 13.04.2016 16 Uhr bei Darren AG Koordination
\end{enumerate} 
Eliah
\begin{enumerate}
	\item 2016-04-11: Mittwoch, 13.04.2016 16 Uhr bei Darren AG Koordination
\end{enumerate}
Esteban
\begin{itemize}
	\item 2016-04-11: Mittwoch, 13.04.2016 16 Uhr bei Darren AG Koordination
\end{itemize}
Gudrun
\begin{enumerate}
	\item 2016-04-04: Bluebox Fairteiler Kontakt an Esteban schicken 
\end{enumerate}
Malou
\begin{enumerate}
    \item 2016-04-04: Flyer zum veganen Brunch mitnehmen
\end{enumerate}
Malte
\begin{enumerate}
	\item 2016-04-04: Angebot Mensa beflyern durchführen.
\end{enumerate}
Philip E.
\begin{enumerate}
    \item 2016-03-14: Philip E-Mail von Dörthe an Benita
    \item 2016-04-04: Schichtleitung Infostand am Infostand.
    \item 2016-04-11: Raphael Fellmer Vortrag mit allem drum und dran organisieren (und delegieren).
\end{enumerate}
Steffi
\begin{enumerate}
	 \item 2016-04-11: Bewerbung für ein Lastenrad
\end{enumerate}
Tobi
\begin{itemize}
	\item 2016-04-11: Mittwoch, 13.04.2016 16 Uhr bei Darren AG Koordination
\end{itemize}

\chapter{Anhang}

\section{Selber versichern}\label{app:versichern}
\includepdf[pages=2]{2016-03-09_foodsharing_versicherung}    

\section{Buttons gemacht}
In einer selbstlosen Aktion von Becci, Malou und Steffi wurde ein Haufen Buttons gemacht. Vielen lieben Dank.\\
\begin{center}
\includegraphics[angle=-90,width=0.5\textwidth]{photo_2016-04-13_06-19-42}
\end{center}

\end{document}

